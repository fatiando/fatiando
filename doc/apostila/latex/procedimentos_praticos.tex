\chapter{Procedimentos práticos}
\label{chap:proc_praticos}

Durante a solução de problemas inversos nos deparamos com diversos obstáculos:

\begin{itemize}
\item Qual a influência que os ruídos aleatórios presentes nos dados exercem sobre o resultado?
\item Por que a inversão não foi capaz de ajustar os dados observados?
\item Como determinar uma valor adequado para o parâmetro de regularização?
\item Como analisar a estabilidade da solução?
\end{itemize}

\noindent Neste capítulo descreveremos alguns procedimentos práticos para abordar
estas questões.

\section{Ajustar os dados observados}

em situações reais, nem sempre é possível ajustar os dados observados
isso pode acontecer, basicamente, por três razões:

i) Os dados observados sempre contêm ruído;
ii) A função escolhida para descrever a relação entre os parâmetros e os dados preditos não é
capaz de descrever o sistema físico de forma satisfatória e
iii) Regularização.

em situações reais, os dados observados sempre contem algum tipo de ruído e este é
impossível de ser removido completamente
na maioria das vezes, a função escolhida para descrever a relação entre os parâmetros e os
dados preditos descreve o sistema físico de maneira aproximada, mas não descreve o ruído,
uma vez que este é, geralmente, de caráter aleatório
sendo assim, a presença de ruído contribui para que os parâmetros estimados no problema
inverso não ajustem perfeitamente os dados observados
há situações em que a função escolhida para descrever a relação entre os parâmetros e os
dados preditos não representa o sistema físico de maneira satisfatória
nesse caso, mesmo se os dados observados não contivessem ruído (situação impossível em
condições reais), não seria possível estimar parâmetros que ajustassem perfeitamente os
dados observados
o fato de uma determinada relação funcional não ser capaz de ajustar os dados observados
significa que a hipótese escolhida para representar o sistema físico não é válida e, portanto, é
um resultado tão importante quanto encontrar uma hipótese que descreva o sistema físico de
forma satisfatória.


\section{Determinar um valor para o parâmetro de regularização}

ressaltar que: se apertar o mi, regulariza(caso a função regularizadora seja certa) e desajusta
os dados; se afroxa mi, ajusta o dado (caso f seja certa) e deixa o problema cagada

uma vez definida a informação a priori a ser introduzida no problema e, consequentemente,
a função regularizadora a ser utilizada, é necessário estabelecer o valor do parâmetro de
regularização que torna o problema inverso bem-posto.
embora exista alguns métodos para escolha do parâmetro de regularização (ASTER;
BORCHERS; THURBER, 2005), ainda não há uma maneira de determinar um valor ótimo
Sendo assim, vamos apresentar um procedimento prático que geralmente é empregado em
situações reais. Este procedimento consiste em escolher o menor parâmetro de regularização
possível para que o problema inverso seja bem-posto e está exemplificado abaixo:
Seja um conjunto de N dados observados.

1) gerar um vetor de valores aleatórios com elementos definidos por realizações de uma
variável aleatória com distribuição de probabilidade Gaussiana, média nula e desvio padrão
igual ao nível de ruído dos dados (que é uma informação a priori).
2) gerar um vetor de dados observados perturbados a partir da soma entre o vetor de valores
aleatórios e o vetor de dados observados originais
3) repetir as etapas 1 e 2 para gerar um conjunto de Q vetores de dados observados
perturbados diferentes
4) estabelecer um valor pequeno para o parâmetro de regularização mi (Equação 26)
5) estimar Q vetores de parâmetros diferentes, sendo um para cada vetor de dados observados
perturbados
6) utilizando o conjunto de Q vetores de parâmetros estimados, calcular a média e o desvio
padrão de cada parâmetro
7) se nenhum parâmetro apresentar um desvio padrão maior que um valor pré estabelecido,
considere que o valor do parâmetro de regularização mi é adequado e o problema inverso está
bem regularizado
8) se ao menos um dos parâmetros apresentar um desvio padrão maior que um valor pré
estabelecido, considere que o valor do parâmetro de regularização mi não é adequado e o
problema inverso não está bem regularizado. Nesse caso, aumente o valor do parâmetro de
regularização mi e repita as etapas 5 a 8.
