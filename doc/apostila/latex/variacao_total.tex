\section{Variação total}

Ao contrário do que vimos na Seção \ref{sec:smoothness}, há situações em que é
desejável que hajam algumas {\it discontinuidades} entre parâmetros espacialmente
adjacentes. Nestes casos, podemos utilizar a função regularizadora de
{\it variação total}

\begin{equation}
\theta^{VT}(\vect{p}) = \sum\limits_{k=1}^L |q_k| \thinspace ,
\end{equation}

\noindent em que $q_k$ é o $k$-ésimo elemento do vetor de diferenças $\vect{q}$
(equação \ref{eq:vetor_diferencas}) e $L$ é o número de elementos em $\vect{q}$.
Como esta função não é diferenciável para valores de $q_k = 0$, podemos
aproximá-la por

\begin{equation}
\theta^{VT}(\vect{p}) \approx \theta^{VT}_\beta(\vect{p}) =
    \sum\limits_{k=1}^L \sqrt{q_k^2 + \beta} \thinspace ,
\end{equation}

\noindent sendo $\beta$ um escalar positivo pequeno.
\\
\indent O vetor gradiente $\vect{\nabla}\theta^{VT}_\beta(\vect{p})$ e a matriz
Hessiana $\mat{\nabla}\theta^{VT}_\beta(\vect{p})$ desta função
\citep{martins_etal2011} são, respectivamente,
\begin{equation}
\vect{\nabla}\theta^{VT}_\beta(\vect{p}) = 2\mat{R}^T \mat{R}\thinspace\vect{p}
\end{equation}

\noindent e

\begin{equation}
\mat{\nabla}\theta^{VT}_\beta(\vect{p}) = 2\mat{R}^T \mat{R} \thinspace .
\end{equation}
