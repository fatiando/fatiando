\section{Variação total}

Ao contrário do que vimos na Seção \ref{sec:smoothness}, há situações em que é
desejável que hajam algumas {\it descontinuidades} entre parâmetros espacialmente
adjacentes. Nestes casos, podemos utilizar a função regularizadora de
{\it variação total}

\begin{equation}
\theta^{VT}(\vect{p}) = \sum\limits_{k=1}^L |v_k| \thinspace ,
\end{equation}

\noindent em que $v_k$, $k=1,2,\dotsc,L$, é o $k$-ésimo elemento do vetor de
diferenças $\vect{v}$ (equação \ref{eq:vetor_diferencas}).
Como esta função não é diferenciável para valores de $v_k = 0$, podemos
aproximá-la por

\begin{equation}
\theta^{VT}(\vect{p}) \approx \theta^{VT}_\beta(\vect{p}) =
    \sum\limits_{k=1}^L \sqrt{v_k^2 + \beta} \thinspace ,
\end{equation}

\noindent sendo $\beta$ um escalar positivo pequeno.
\\
\indent O vetor gradiente $\vect{\nabla}\theta^{VT}_\beta(\vect{p})$ e a matriz
Hessiana $\mat{\nabla}\theta^{VT}_\beta(\vect{p})$ desta função
são, respectivamente, \citep{martins_etal2011}

\begin{equation}
\vect{\nabla}\theta^{VT}_\beta(\vect{p}) = \mat{R}^T \vect{q}(\vect{p})
\label{eq:grad_tv}
\end{equation}

\noindent e

\begin{equation}
\mat{\nabla}\theta^{VT}_\beta(\vect{p}) = \mat{R}^T \mat{Q}(\vect{p})\mat{R}
\thinspace ,
\label{eq:hessian_tv}
\end{equation}

\noindent
sendo $\mat{R}$ uma matriz de diferenças finitas
(equação \ref{eq:vetor_diferencas}).
O vetor $\vect{q}(\vect{p})$ e a matriz $\mat{Q}(\vect{p})$ são, respectivamente,

\begin{equation}
\vect{q}(\vect{p}) =
    \begin{bmatrix}
    \dfrac{v_1}{\sqrt{v_1^2 + \beta}} \vspace{0.3cm}\\
    \dfrac{v_2}{\sqrt{v_2^2 + \beta}} \\
    \vdots \\ \dfrac{v_L}{\sqrt{v_L^2 + \beta}}
    \end{bmatrix} \thinspace
\end{equation}

\noindent e

\begin{equation}
\mat{Q}(\vect{p}) =
    \begin{bmatrix}
    \dfrac{\beta}{(v_1^2 + \beta)^{\frac{3}{2}}} & 0 & \ldots & 0 \vspace{0.3cm}\\
    0 & \dfrac{\beta}{(v_2^2 + \beta)^{\frac{3}{2}}} & \ldots & 0 \\
    \vdots & \vdots & \ddots & \vdots \\
    0 & 0 & \ldots & \dfrac{\beta}{(v_L^2 + \beta)^{\frac{3}{2}}}
    \end{bmatrix} \thinspace .
\end{equation}

\indent Como o gradiente e a Hessiana da função regularizadora de variação
total (equações \ref{eq:grad_tv} e \ref{eq:hessian_tv}) não são combinações
lineares dos parâmetros, a utilização desta função regularizadora torna o
problema inverso {\it não-linear}. Assim sendo, a equação normal do
{\it problema inverso não-linear regularizado}, para o caso da regularização de
variação total, é

\begin{equation}
\left[\mat{G}(\vect{p}_0)^T\mat{G}(\vect{p}_0) +
      \mu\mat{R}^T \mat{Q}(\vect{p}_0)\mat{R}\right]\Delta\vect{p} =
\mat{G}(\vect{p}_0)^T \left[\vect{d}^{\thinspace o} - \vect{f}(\vect{p}_0)\right] -
\mu\mat{R}^T \vect{q}(\vect{p}_0)
    \thinspace .
\end{equation}

\noindent em que $\vect{f}(\vect{p}_0)$ é o vetor de dados preditos avaliado em
$\vect{p}_0$ e $\Delta\vect{p}$ é a correção a ser aplicada a $\vect{p}_0$.


