\chapter{Regularização}

As equações \ref{eq:sistema_normal} do problema inverso linear e
\ref{eq:sistema_normal_nlin} do problema inverso não-linear são
{\it equações normais}.
Estas equações são sistemas lineares cuja matriz é quadrada.
Para que a solução de uma equação normal seja única, é necessário que esta
matriz tenha {\it posto completo}.
Uma condição para que uma matriz tenha posto completo é que todas as suas colunas
(ou linhas) sejam {\it linearmente independentes}.
Esta condição é equivalente a dizer que a matriz possua determinante diferente de
zero.
\\
\indent Em problemas geofísicos, é comum que a matriz do sistema normal possua
determinante próximo de zero. Ou seja, para fins práticos pode-se considerar que
a matriz {\it não} tenha posto completo.
Isto faz com que o problema inverso geofísico seja um problema {\it mal posto}.

\begin{quote}
{\tt Um problema {\bf mal-posto} apresenta, principalmente, {\it instabilidade}
e {\it falta de unicidade} da solução.}
\end{quote}

\indent A instabilidade é definida como

\begin{quote}
{\tt A {\bf instabilidade} é a alta variabilidade dos parâmetros perante
peque\-nas variações nos dados.}
\end{quote}

\noindent Por exemplo, seja um conjunto de dados preditos A e outro conjunto de
dados preditos ligeiramente diferentes B.
Se o problema inverso apresenta instabilidade, os pa\-râ\-me\-tros que produzem os
dados preditos A são consideravelmente diferentes daqueles que produzem os dados
preditos B.

meh

\section{Norma mínima (Tikhonov de ordem 0)}

A função regularizadora mais comumente usada é a chamada {\it norma mínima}
(também conhecida como {\it ridge regression} ou {\it Tikhonov de ordem zero}).
Como seu nome sugere, esta função é utilizada para incorporar a informação de
que o vetor de parâmetros deve ter a norma quadrática ($\ell_2$ ou Euclidiana)
mínima.
Isto é, os parâmetros devem ser assumir valores mais próximos possíveis a zero.
De forma similar a equação \ref{eq:dotprod}, a função regularizadora de norma
mínima tem a seguinte forma

\begin{equation}
\theta^{NM}(\vect{p}) = \vect{p}^T\vect{p} \thinspace .
\label{eq:norma_minima}
\end{equation}

\indent O vetor gradiente $\vect{\nabla}\theta^{NM}(\vect{p})$ e a matriz Hessiana
$\mat{\nabla}\theta^{NM}(\vect{p})$ (ver Apêndice \ref{chap:opmat}) desta função
são, respectivamente,

\begin{equation}
\vect{\nabla}\theta^{NM}(\vect{p}) = 2\mat{I}\vect{p}
\label{eq:grad_norma_minima}
\end{equation}

\noindent e

\begin{equation}
\mat{\nabla}\theta^{NM}(\vect{p}) = 2\mat{I} \thinspace ,
\label{eq:hessian_norma_minima}
\end{equation}

\noindent em que $\mat{I}$ é a matriz identidade de dimensão $M \times M$,
lembrando que $M$ é o número de parâmetros.
Note que o gradiente da função regularizadora de norma mínima é uma
{\it combinação linear dos parâmetros}.
\\
\indent Para o caso em que a função $f_i(\vect{p})$ que relaciona
os dados preditos aos parâmetros também é {\it linear} (equação \ref{eq:comb_linear}),
a equação normal do {\it problema inverso linear regularizado},
para o caso da regularização de norma mínima, é

\begin{equation}
\left(\mat{G}^T\mat{G} + \mu\mat{I}\right)\opt{p} =
    \mat{G}^T\left(\vect{d}^{\thinspace o} - \vect{b} \right) ,
\label{eq:sistema_normal_norma_min_linear}
\end{equation}

\noindent em que $\mu$ é o parâmetro de regularização, $\vect{d}^{\thinspace o}$
é o vetor de dados observados, $\mat{G}$ é a matriz de sensibilidade, $\vect{b}$
é um vetor de constantes (equação \ref{eq:f_igual_Gp}) e $\opt{p}$ é a solução
de norma mínima para o problema inverso linear.
\footnote{
Onde foi parar o termo $\vect{\nabla}\theta^{NM}(\vect{p}_0)$? Dica: Mostre que, se o
problema inverso é linear, a estimativa $\opt{p}$ não depende da aproximação
inicial $\vect{p}_0$.}
\\
\indent Já para o caso em que $f_i(\vect{p})$ é {\it não-linear}, o problema
inverso torna-se também não-linear. Assim sendo, a equação normal do
{\it problema inverso não-linear regularizado}, para o caso da regularização de
norma mínima, é

\begin{equation}
\left[\mat{G}(\vect{p}_0)^T\mat{G}(\vect{p}_0) +
      \mu\mat{I}\right]\Delta\vect{p} =
\mat{G}(\vect{p}_0)^T \left[\vect{d}^{\thinspace o} - \vect{f}(\vect{p}_0)\right] -
\mu\vect{p}_0
    \thinspace .
\label{eq:sistema_normal_norma_min_naolinear}
\end{equation}

\noindent em que $\vect{f}(\vect{p}_0)$ é o vetor de dados preditos avaliado em
$\vect{p}_0$ e $\Delta\vect{p}$ é a correção a ser aplicada a $\vect{p}_0$.

\section{Suavidade (Tikhonov de order 1)}

meh
bla

\section{Igualdade}

Há casos em que se conhece um valor aproximado de um ou mais parâmetros.
Esta informação pode ser proveniente de furos de sondagem, afloramentos, etc.
Nestes casos, é desejável que o valor estimado para estes parâmetros seja o mais
próximo possível dos valores conhecidos (e preestabelecidos).
Para tanto, utilizamos a {\it função regularizadora de igualdade}

\begin{equation}
\theta(\vect{p}) =
    \left(\vect{p} - \vect{p}^{\thinspace a}\right)^T \mat{A}
        \left(\vect{p} - \vect{p}^{\thinspace a} \right)
    \thinspace ,
\label{eq:igualdade}
\end{equation}

\noindent em que $\vect{p}^{\thinspace a}$ é um vetor cujo $j$-ésimo elemento é
ou o valor conhecido (preestabelecido) do $j$-ésimo parâmetro, ou zero caso não
haja um valor conhecido do $j$-ésimo parâmetro. A matriz $\mat{A}$ é uma matriz
diagonal quadrada de dimensão $M \times M$, lembrando que $M$ é o número de
parâmetros.
O $j$-ésimo elemento da diagonal de $\mat{A}$ é 1 (um) caso
haja um valor conhecido (preestabelecido) para $j$-ésimo parâmetro, ou 0 (zero)
caso contrário.
\\
\indent Por exemplo, digamos que em um determinado problema inverso a função
$f_i(\vect{p})$ depende de três parâmetros. Além disso, desejamos que o segundo
parâmetro $p_2$ seja o mais próximo possível do valor $26$.
Podemos então usar a função regularizadora de igualdade para impor essa restrição.
Neste caso, os vetores $\vect{p}$ e $\vect{p}^{\thinspace a}$ e a matriz
$\mat{A}$ serão

\[
\vect{p} =
    \begin{bmatrix}
    p_1 \\ p_2 \\ p3
    \end{bmatrix} , \quad
\vect{p}^{\thinspace a} =
    \begin{bmatrix}
    0 \\ 26 \\ 0
    \end{bmatrix} \quad \text{e} \quad
\mat{A} = 
    \begin{bmatrix}
    0 & 0 & 0 \\
    0 & 1 & 0 \\
    0 & 0 & 0
    \end{bmatrix} .
\]


\indent O vetor gradiente $\vect{\nabla}\theta(\vect{p})$ e a matriz Hessiana
$\mat{\nabla}\theta(\vect{p})$ (ver Apêndice \ref{chap:opmat}) da função
regularizadora de igualdade são, respectivamente,

\begin{equation}
\vect{\nabla}\theta(\vect{p}) = 2\mat{A}
    \left(\vect{p} - \vect{p}^{\thinspace a} \right)
\end{equation}

\noindent e

\begin{equation}
\mat{\nabla}\theta(\vect{p}) = 2\mat{A} \thinspace .
\end{equation}

\indent Para o caso em que a $f_i(\vect{p})$ que relaciona
os dados preditos aos parâmetros também é {\it linear} (equação \ref{eq:comb_linear}),
a equação normal do {\it problema inverso linear regularizado},
para o caso da regularização de igualdade, é

\begin{equation}
\left(\mat{G}^T\mat{G} + \mu\mat{A}\right)\opt{p} =
    \mat{G}^T\left(\vect{d}^{\thinspace o} - \vect{b} \right) +
    \mu\mat{A}\vect{p}^{\thinspace a} ,
\end{equation}

\noindent em que $\mu$ é o parâmetro de regularização, $\vect{d}^{\thinspace o}$
é o vetor de dados observados, $\mat{G}$ é a matriz de sensibilidade, $\vect{b}$
é um vetor de constantes (equação \ref{eq:f_igual_Gp}) e $\opt{p}$ é a solução
do problema inverso linear com vínculos de igualdade.
\\
\indent Já para o caso em que $f_i(\vect{p})$ é {\it não-linear}, o problema
inverso torna-se também não-linear. Assim sendo, a equação normal do
{\it problema inverso não-linear regularizado}, para o caso da regularização de
igualdade, é

\begin{equation}
\left[\mat{G}(\vect{p}_0)^T\mat{G}(\vect{p}_0) +
      \mu\mat{A}\right]\Delta\vect{p} =
\mat{G}(\vect{p}_0)^T \left[\vect{d}^{\thinspace o} - \vect{f}(\vect{p}_0)\right] -
\mu\mat{A}\left[\vect{p}_0 - \vect{p}^{\thinspace a}\right]
    \thinspace .
\end{equation}

\noindent em que $\vect{f}(\vect{p}_0)$ é o vetor de dados preditos avaliado em
$\vect{p}_0$ e $\Delta\vect{p}$ é a correção a ser aplicada a $\vect{p}_0$.


\section{Variação total}

Ao contrário do que vimos na Seção \ref{sec:smoothness}, há situações em que é
desejável que hajam algumas {\it discontinuidades} entre parâmetros espacialmente
adjacentes. Nestes casos, podemos utilizar a função regularizadora de
{\it variação total}

\begin{equation}
\theta^{VT}(\vect{p}) = \sum\limits_{k=1}^L |q_k| \thinspace ,
\end{equation}

\noindent em que $q_k$ é o $k$-ésimo elemento do vetor de diferenças $\vect{q}$
(equação \ref{eq:vetor_diferencas}) e $L$ é o número de elementos em $\vect{q}$.
Como esta função não é diferenciável para valores de $q_k = 0$, podemos
aproximá-la por

\begin{equation}
\theta^{VT}(\vect{p}) \approx \theta^{VT}_\beta(\vect{p}) =
    \sum\limits_{k=1}^L \sqrt{q_k^2 + \beta} \thinspace ,
\end{equation}

\noindent sendo $\beta$ um escalar positivo pequeno.
\\
\indent O vetor gradiente $\vect{\nabla}\theta^{VT}_\beta(\vect{p})$ e a matriz
Hessiana $\mat{\nabla}\theta^{VT}_\beta(\vect{p})$ desta função
\citep{martins_etal2011} são, respectivamente,
\begin{equation}
\vect{\nabla}\theta^{VT}_\beta(\vect{p}) = 2\mat{R}^T \mat{R}\thinspace\vect{p}
\end{equation}

\noindent e

\begin{equation}
\mat{\nabla}\theta^{VT}_\beta(\vect{p}) = 2\mat{R}^T \mat{R} \thinspace .
\end{equation}

\input{latex/compacidade.tex}
